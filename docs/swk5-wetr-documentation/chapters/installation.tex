\section{Installationsanleitung}

\subsection{Benötigte Programme und Voraussetzungen}
Für dieses Projekt wird zusätzliche Software benötigt:
\begin{itemize}
\item Docker \footnote{https://www.docker.com/}
\item Visual Studio \footnote{https://visualstudio.microsoft.com/de/}
\item MySql-Connector (optional, nur falls notwendig) \footnote{https://dev.mysql.com/get/Downloads/Connector-Net/mysql-connector-net-8.0.13.msi}
\end{itemize}
\raggedright

Um die Datenbank zu erstellen muss Docker gestartet sein. Die Datenbank kann mit dem \textit{Powershell-Script} \grqq{}run.ps1 \grqq{} automatisch generiert werden.
Falls dieses Script wegen fehlender Berechtigungen nicht ausgeführt werden kann, muss eine \textit{Shell} (Git-Bash oder ähnliches) im Ordner mit der Docker Compose Datei \grqq{}docker-compose.yaml \grqq{} geöffnet und folgenden Befehle nacheinander ausgeführt werden:

\begin{minted}{dockerfile}
docker stop $(docker ps -a -q)
docker rm $(docker ps -a -q)
docker-compose up --build --force-recreate
\end{minted}

\subsection{Datenbank}
Für dieses Projekt werden zwei Datenbanken benötigt, wobei die zweite eine Unit-Test Datenbank ist, die nur zur Ausführung solcher Tests benötigt wird.\\

Alle SQL Skripts müssen im \textit{PhpMyAdmin} Interface unter dem \textit{Import} Tab ausgeführt werden. \textit{PhpMyAdmin} ist unter der Addresse \grqq{}\textit{http://localhost:8080/}\grqq{} erreichbar.\\

Die benötigten Datenbanken können mit Hilfe von den zwei Skripts \glqq create\_wetr.sql\grqq\ und \glqq create\_wetr-unit-testing.sql\grqq\ ,welche sich im sql/Create Ordner befinden, erstellt werden. 
\newline\newline
Nach Ausführen der Create-Scripts können für die Produktivdatenbank (wetr) Beispieldaten eingefügt werden. Dazu die Datenbank ausfählen und wieder in den Import Tab wechseln. Die zum Einfügen benötigte Datei \glqq InsertEverythingWithoutMeasurement.sql \grqq\ befindet sich im sql/Insert Ordner und fügt alle Beispieldaten, außer die der Messwerte, in die Datenbank ein.\\

Die Messdaten müssen zuerst generiert werden. Hierzu muss die Datei \glqq Wetr.Generator.exe \grqq\ ausgeführt werden. Danach sollten sich sechs \textit{.bulk} Datein im Verzeichnis befinden, welche alle Messdaten für die verschiednen Kategorien beinhalten. Um die Daten in die Datenbank einzufügen müssen die Befehle im Skript \glqq InsertMeasurements.sql \grqq\ in PhpMyAdmin ausgeführt werden.\\

\newpage
\begin{minted}{sql}
LOAD DATA LOCAL INFILE "/tmp/sql/insert/measurementsDownfall.bulk" 
    INTO TABLE measurement
FIELDS TERMINATED BY ', ' ENCLOSED BY "'"
LINES TERMINATED BY '\r\n';

LOAD DATA LOCAL INFILE "/tmp/sql/insert/measurementsHumidity.bulk" 
    INTO TABLE measurement
FIELDS TERMINATED BY ', ' ENCLOSED BY "'"
LINES TERMINATED BY '\r\n';

LOAD DATA LOCAL INFILE "/tmp/sql/insert/measurementsTemperature.bulk" 
    INTO TABLE measurement
FIELDS TERMINATED BY ', ' ENCLOSED BY "'"
LINES TERMINATED BY '\r\n';

LOAD DATA LOCAL INFILE "/tmp/sql/insert/measurementsWind.bulk" 
    INTO TABLE measurement
FIELDS TERMINATED BY ', ' ENCLOSED BY "'"
LINES TERMINATED BY '\r\n';

LOAD DATA LOCAL INFILE "/tmp/sql/insert/measurementsWindDirection.bulk" 
    INTO TABLE measurement
FIELDS TERMINATED BY ', ' ENCLOSED BY "'"
LINES TERMINATED BY '\r\n';

LOAD DATA LOCAL INFILE "/tmp/sql/insert/measurementsPreassure.bulk" 
    INTO TABLE measurement
FIELDS TERMINATED BY ', ' ENCLOSED BY "'"
LINES TERMINATED BY '\r\n';
\end{minted}